\documentclass{kuisthesis}

\jtitle[スマートコントラクトのガス消費量のReource Aware MLを\\用いた静的解析]
  {スマートコントラクトのガス消費量の\\Reource Aware MLを用いた静的解析}
\etitle{Static Analysis for Gas Consumption of Smart Contracts Using Resource Aware ML}
\jauthor{小野 雄登}
\eauthor{Yuto Ono}
\supervisor{末永 幸平 教授}
\date{2021年2月2日}

\begin{document}
\maketitle

\begin{jabstract}
2009年にビットコインを用いた取引がオープンソフトウェアで始まって以来,
現在に至るまでにブロックチェーンを技術基盤とする様々な仮想通貨が開発されている.
スマートコントラクトは,仮想通貨の取引における契約の締結や履行を自動化する仕組みであり,
ブロックチェーン上で動作するプログラムとして実装される.

スマートコントラクトにはガスの概念が存在し,ガスはコントラクトの実行にかかる手数料を表している.
コントラクトが実行される際に,コントラクトの各命令の評価毎に命令の内容に比例した量のガスが消費され,
消費量の合計が許容ガス消費量を超えると,その命令が直ちに停止され,命令の実行による値の変更が取り消される.
プログラムとして非効率なコントラクトが実行されると,想定される量以上のガスが消費されてしまうので,
コントラクトのガス消費量を静的に解析することは,
ユーザーが必要以上に手数料を支払わないために必要な技術であると考えられる.

本研究では,スマートコントラクトのガス消費量の静的な解析を行うプログラムを実装する.
具体的には,スマートコントラクトを実装しているブロックチェーンプロトコルであるTezosにおいて,
スタックベースのプログラミング言語Michelsonで書かれたコントラクトのガス消費量を,
プログラミング言語型のツールであるResource Aware ML(RAML)を用いて静的に解析する.
RAMLは,OCamlで用いられる文法を備えた関数型プログラミング言語で,
入力として与えられたプログラムのリソース消費量の上界を,
指定されたメトリックに従って自動的に,かつ静的に解析して,その結果を出力するツールである.

実装の方針は,
\begin{enumerate}
  \item Michelsonの挙動を再現するためのライブラリをRAMLで設計する.
  \item 設計したライブラリを用いて,Michelsonで書かれたコントラクトをエンコードする.
  \item エンコードしたコントラクトを解析し,ガス消費量を見積もる.
\end{enumerate}
という流れである.以下,各過程について説明する.


1.において,Michelsonはスタック構造をもち,
コントラクトに用いられる命令は,初期スタックを受け取ってスタックの内容を変更して返す関数として実装されている.
この構造をRAMLで設計するにあたって,スタックの要素をヴァリアント型tとして定義し,
命令を(t list -> t list)型をもつ関数として定義した.

2.において,Michelsonのコントラクトは,初期スタックに入る値の型宣言と,
初期スタックに対して順に適用される一連の命令によって構成されている.
RAMLにおいてこのコントラクトを,1.で設計したライブラリを用いて,
初期スタックを表すリストに対して命令を順に関数適用するプログラムとして実装した.

3.において,2.で実装したプログラムを,RAMLのstepsメトリックを用いて評価ステップ数に関する解析を行った結果,
基本的なスタックの操作に関する命令や,簡単な算術演算や条件分岐の命令のみを含むコントラクトについては解析が正しく行われたが,
ループ命令や,リストや集合に対する再帰を行う命令を含むコントラクトについては解析が失敗するものも存在した.
続いて,ガス消費量の見積もりについては,RAMLのtickメトリックを用いた.
tickメトリックは,リソース消費の値や発生するタイミングを,ユーザーが関数として定義することができるメトリックである.
RAMLで実装した各命令について,その命令のガス消費量に相当する値のtick関数を定義し,tickメトリックを用いた解析を行い,
コントラクトのガス消費量を見積もれるかどうかを検証した.
結果として,コントラクトの実行において発生するガス消費のうち,
プログラムの解釈実行を行う際に発生するinterpreter costについて概ね正しく消費量を見積もることができた.

本研究においては,Michelsonに実装されているコントラクトの命令のうち,主要なものについてRAMLで実装した.
残りの命令の実装については,今後の課題とする.
また,ガス消費量の見積もりについてはinterpreter costについてのみ取り組んだが,他の過程において発生するガス消費量の見積もり,
ひいてはコントラクトの実行において発生するガス消費量全体の見積もりについても検討していきたい.

\end{jabstract}

\begin{eabstract}

\end{eabstract}

\tableofcontents

\section{序論}\label{sec-intro}
2009年にビットコインを用いた取引がオープンソフトウェアで始まって以来,
現在に至るまでにブロックチェーンを技術基盤とする様々な仮想通貨が開発されている.
取引の記録をブロックとしてネットワーク上に記憶するという性質上,ブロックチェーンはデータ改竄に対する優れた耐性を持ち,
仮想通貨の取引を支えるコア技術となっている.

ブロックチェーン上で用いられる技術としてスマートコントラクトがある.
スマートコントラクトは,仮想通貨の取引における契約の締結や履行を自動化する仕組みであり,
ブロックチェーン上で動作するプログラムとして扱われる.
第3者を介さずに,また相手の信頼を必要とせずに取引を行うことができ,決済期間の短縮や手数料の削減などの効果が期待できる.
スマートコントラクトにはガスの概念が存在し,ガスはコントラクトの実行にかかる手数料を表している.
コントラクトが実行される際に,コントラクトの各命令の評価毎に命令の内容に比例した量のガスが消費され,
消費量の合計が許容ガス消費量を超えると,その命令が直ちに停止され,命令の実行による値の変更が取り消される.
ガスの消費量の計算は複雑で,前もってガスの消費量を正確に見積もることは難しいとされているが,
プログラムとして非効率なコントラクトが実行されると,想定される量以上のガスが消費されてしまうので,
コントラクトのガス消費量を静的に解析することは,
ユーザーが必要以上に手数料を支払わないために必要な技術であると考えられる.

本研究では,仮想通貨Tezosのスマートコントラクトのガス消費量の静的な解析を行うプログラムを実装した.
Tezosはスマートコントラクトを用いたブロックチェーンを技術基盤とする仮想通貨の1つで,
コントラクトはスタックベースのプログラミング言語Michelsonで書かれている.
コントラクトのガス消費量はMichelsonプログラムの実行内容によって計算されるので,
このプログラムに対して解析を行うことでガス消費量の見積もりを試みた.

解析の方法として,プログラミング言語型のツールであるResource Aware ML(RAML)を用いる.
RAMLは,OCamlで用いられる文法を備えた関数型プログラミング言語で,
入力として与えられたプログラムのリソース消費量の上界を,
指定されたメトリックに従って自動的に,かつ静的に解析して,その結果を出力するツールである.
具体的な方法としては,Michelsonで実装されている型や命令などのライブラリをRAMLで実装し,
MichelsonのコントラクトをRAML上でエンコードし,それを解析してガス消費量の見積もる.

本報告書は以下のように構成されている.第2章では,本研究の背景知識として,TezosとMichelsonプログラム,
スマートコントラクトにおけるガス消費の仕組み,そして解析に用いるツールであるRAMLについてそれぞれ説明する.
第3章では,MihcelsonプログラムのRAMLでの実装について説明する.
第4章では,第3章で実装したRAMLプログラムを用いて行ったガス消費量の解析について,結果と考察を記述する.
第5章では,実装したプログラムについていくつかの改善点を示す.
最後に第6章で本研究についての結論を述べる.


\section{背景知識}\label{sec-preliminary}
本章では,


\subsection{TezosとMichelsonについて}\label{subsec-pre-tezos}

\subsection{コントラクトのガス消費の仕組み}\label{subsec-pre-gas}

\subsection{Resource Aware MLについて}\label{subsec-pre-raml}

\section{RAMLでのMichelsonプログラムの実装}

\section{ガス消費量の解析の結果と考察}

\section{改善点}

\section{結論}\label{sec-conclusion}

\acknowledgments

\nocite{*}
\bibliographystyle{kuisunsrt}
\bibliography{main}

\end{document}