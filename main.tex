\documentclass{kuisthesis}

\jtitle[スマートコントラクトのガス消費量のReource Aware MLを\\用いた静的解析]
  {スマートコントラクトのガス消費量の\\Reource Aware MLを用いた静的解析}
\etitle{Static Analysis for Gas Consumption of Smart Contracts Using Resource Aware ML}
\jauthor{小野 雄登}
\eauthor{Yuto Ono}
\supervisor{末永 幸平 教授,五十嵐 淳 教授}
\date{2021年2月2日}

\begin{document}
\maketitle

\begin{jabstract}
2009年にビットコインを用いた取引がオープンソフトウェアで始まって以来,
現在に至るまでにブロックチェーンを技術基盤とする様々な仮想通貨が開発されている.
スマートコントラクトは,仮想通貨の取引における契約の締結や履行を自動化する仕組みであり,
ブロックチェーン上で動作するプログラムとして扱われる.

Tezosは,スマートコントラクトを用いたブロックチェーンを技術基盤とする仮想通貨の1つで,
コントラクトはスタックベースのプログラミング言語Michelsonで書かれている.
Tezosのスマートコントラクトには,ガスの概念が存在し,
ガスはコントラクトの実行にかかる手数料を表している.
コントラクトの各命令のの評価毎に命令の実行内容に比例した量のガスが消費され,
消費量の合計が許容ガス消費量を超えると,その命令が直ちに停止され,命令の実行による値の変更が取り消される.
ガスの消費量の計算は複雑で,前もってガスの消費量を正確に見積もることは難しいとされている.

本研究では,プログラミング言語型のツールであるResource Aware ML(以下,RAMLと略記する.)を用いて,
スマートコントラクトのガス消費量の静的な解析を行うプログラムを実装した.
RAMLは,OCamlの文法を用いたプログラムを入力として受け取り,
プログラムのリソース消費量の限界値を指定されたメトリックに従って自動的に,かつ静的に計算し,
その解析結果を多項式の値として出力するツールである.

実装の過程は大きく2つに分けられる.
始めに,Michelsonプログラムのスタック構造を,RAMLにおいてOcamlのリスト構造を用いたプログラムとして実装した.
次に,実装したプログラムに対して,RAMLのメトリックの1つであるtickメトリックを用いて,
出力として得られるプログラムのリソース消費量からコントラクトのガス消費量を見積もれるかどうかを検証した.
tickメトリックは,リソース消費の値や発生するタイミングを,ユーザーが関数として定義することができるメトリックである.

MihcelsonプログラムのRAMLでの実装では,スタックの要素をヴァリアント型tとして定義し,
コントラクトの各命令を(t list -> t list)型をもつ関数として定義した.
Michelsonプログラムでのコントラクトは初期スタックとそれを変更する一連の命令として実装されているので,
同様にRAMLにおいてコントラクトを実装し,RAMLのstepsメトリックを用いてリソース消費量を計算した.
結果として,基本的なスタックの操作に関する命令や,簡単な算術演算や条件分岐の命令については多項式で表現されたが,
ループ命令や,リストや集合に対する再帰を含んだ命令については多項式で表現できないものも存在した.

ガス消費量の見積もりの検証では,コントラクトの実行におけるガス消費量は,
コントラクトの実行中の複数の過程において発生するガス消費量の合計で計算されているが,
研究における時間的制約から,そのうちの1つであるinterpreter costについて取り組むことに決めた.
RAMLで実装したコントラクトの各命令について,その命令のガス消費量に相当する値のtick関数を定義し,
tickメトリックを用いてリソース消費量の分析を行った.
結果として,コントラクト実行時のログに表示されるinterpreter costに相当する値に概ね等しい値を
出力として得られた.
ただし,ガスの消費量が引数の値に依存するような命令などは,正確な消費量を見積もることはできなかった.

本研究において実装しなかった,もしくはリソース消費量が解析できなかったコントラクト命令の実装や,
コントラクトの実行中の他の過程において発生するガス消費量の見積もりは,今後の課題とする.

\end{jabstract}

\begin{eabstract}

\end{eabstract}

\tableofcontents

\section{序論}\label{sec-intro}
tezosのコントラクトにはガスの概念が用いられている.
コントラクトの実行において一定量のガスが消費されるが,この量が制限を超えるとコントラクトの実行が取り消される.
tezosのコントラクトはスタックベースのプログラミング言語Mihchelsonを用いたプログラムによって書かれており,ガスの消費量はMichelsonプログラムの内容によって決まる.
本研究では,Resource Aware ML(RAML)を用いて,コントラクトのガス消費量の推測を試みる.
RAMLは,OCamlプログラムを入力として受け取り,そのリソース消費量の限界値を出力するプログラミング言語型のツールである.
Michelsonプログラムのスタック構造をOCamlのリスト構造を用いてOCamlプログラムとして表現し,RAMLのtickメトリクスを用いて,
リソース消費量としてコントラクトのガス消費量に近しい値が得られるかどうかを検証する.

\section{背景知識}\label{sec-preliminary}
\subsection{tezosとMichelsonについて}\label{subsec-pre-tezos}

\subsection{コントラクトのガス消費の仕組み}\label{subsec-pre-gas}

\subsection{Resource Aware MLについて}\label{subsec-pre-raml}

\section{RAMLでのMichelsonプログラムの実装}

\section{検証結果と考察}

\section{改善点}

\section{結論}\label{sec-conclusion}

\acknowledgments

\nocite{*}
\bibliographystyle{kuisunsrt}
\bibliography{main}

\end{document}