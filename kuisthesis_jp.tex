\documentclass{kuisthesis}           % 日本語 
%\documentclass[english]{kuisthesis} % 英語

\jtitle[特別研究報告書執筆の手引]%	% 和文題目(内容梗概/目次用)
	{特別研究報告書執筆の手引}	% 和文題目
\etitle{How to Write Your Graduation Thesis}	% 英文題目
\jauthor{末永 幸平}				% 和文著者名
\eauthor{Kohei Suenaga}			% 英文著者名
\supervisor{John Doe 教授}			% 指導教員名
\date{2021年2月10日}				% 提出年月日


\begin{document}
\maketitle					% 「とびら」の出力

\begin{jabstract}				% 和文梗概
この手引は,京都大学工学部情報学科計算機科学コースにおける
特別研究報告書の構成と形式について説明したものである.
また,当コースで定めた形式に則った論文を日本語\LaTeX を用いて作成するための
スタイル・ファイル\verb|kuisthesis|の使い方についても説明している.
なお,この手引自体も\verb|kuisthesis|を用い,
定められた形式に従って作成されているので,
必要に応じてソース・ファイルを参照されたい.
\end{jabstract}

\begin{eabstract}				% 英文梗概
This guide describes how to write your guraduation thesis
according to the regulation of Computer Science Course, 
School of Informatics and Mathematical Science,
Faculty of Engineering Kyoto University. 
This regulation specifies the rules about the structure 
and format of the thesis which you need to follow in writing.
This guide also explains how to use a \LaTeX{} style file for
graduation thesis, named \verb|kuisthesis|, 
with which you can easily produce a well-formatted thesis. 
This guide itself is written using \verb|kuisthesis|; 
the source code may be helpful 
if you would like to know how to use this style file.
\end{eabstract}

\tableofcontents				% 目次の出力

\section{はじめに}\label{sec-intro}		% 本文の開始
特別研究報告書は,学部で行なった研究の成果をまとめて提出するものである.
学部における学修研究の締めくくりとして,また自分の研究成果をある程度の
分量の文章に分かりやすくまとめる経験として,重要な意義を持っている.こ
れらは永く保管され,広く教員,学生の閲覧に供せられることになっている.

この手引では,報告書作成の基本的な要領と形式について,ある程度の指針を
述べている.論文や報告書を執筆する際のスタイルは研究分野によって異なる
ところも多いので,本手引を参考にした上で,指導教員のアドバイスや各々の
研究分野の優れた論文や技術文献を参考にし,またアカデミック・ライティン
グに関する文献も参考にしながら,良い報告書を提出されたい.

手引が示す要領や指示は最低限のものであり,それらに従いさえすればよい論
文ができあがるというものではない.与えられた紙数の枠内で,研究の内容を
簡潔に分かりやすくまとめる能力はなかなか奥が深く,これからの皆さんが人
生をかけて涵養すべき能力である.本報告書の執筆を契機に,日頃から内外の優
れた論文に触れ,自分のライティングの力を高められたい.

% 全体の構成,文章,字句などについて細心
% の注意を払うことが大切である.
%% 記述や表現の能力を養うことが大切である.
%% ことを心がけ,
%% .,各自の研究の成果が読者に正しく理解される
%% ように記述することが最も重要である.このためには,

%% また,

以下,論文の構成と執筆上の注意事項を述べ,付録として論文執筆用の日本語\LaTeX 
スタイルファイル\verb|kuisthesis|の使い方を示す.

\section{報告書の形式について}\label{sec-instruction}
% \subsection{用語}\label{subsec-language}

特別研究報告書は以下の形式に従ってまとめ,別途告知される期日
(厳守:締め切りを過ぎての提出は認められない)までに提出すること.

\begin{itemize}
  \item 特別研究報告書は\emph{日本語}あるいは\emph{英語}で書くこと.
%    計算機科学においては
%    研究成果を英語で発表することが重要となっている一方,日本語で論文を
%    綴る能力も依然として重要であることに鑑み,本コースでは特別研究報告
%    書を日本語で記述することとしている.

  \item 報告書は,A4の用紙の片面に印刷して提出すること.

  \item 報告書には\emph{日本語および英語}の内容梗概を含むこと.
    \emph{それぞれ1ページ半から2ページ}にまとめること.始め
    には,題目および氏名を書くこと.

  % \item 目次の始めにも題目を記載すること.

  \item 報告書本文は25ページ$\pm 10\%$にまとめること.このページ数には
    図表を含み,参考文献は除く.図表の分量は全体の$40\%$程度を限度とし,
    これを超過する場合は適宜付録にまわすこと.
  \item 報告書には,内容梗概のまえに「とびら」をつけ,全体を
    所定のファイルにとじて提出すること.とびらは本文と同じ用紙を用い,
    特別研究報告書である旨,題目,指導教員名,所属学科名,氏名,提出年
    月を記入すること.
  \item 報告書全体はファイルに綴じ,ファイルの表紙にはとびらと同様の事
    項を記載すること.
    
  \item 報告書は手書きではなく,ワープロソフトもしくは\LaTeX{}を用いて
    清書すること.\LaTeX{}を用いる場合は\verb|kuisthesis|スタイルファ
    イルを用いること.ページレイアウトは以下の通りとすること.(配布さ
      れているスタイルファイルを用いれば,これらの要件は満たされるよう
      になっているはずである.)
    \begin{itemize}
    \item 論文の各ページの\emph{左端3\,cmと右端1\,cm}は必ず空白とする
      こと.
    \item 日本語で報告書を執筆する場合は,12\,pt(あるいは相当の大きさ)
      のフォントを用い,\emph{1行当り35 文字, 1ページあたり32行}で製
      版すること.ただし,用いるソフトウェアの都合でこの基準が守れない
      場合は1ページ当りの字数が同程度となるようにして,1行当りの字数や,
      1ページあたりの行数を調整してもよい.
    \item
      英文で報告書を執筆する場合は12\,pt(あるいは相当の大きさ)のフォ
      ントを用い,\emph{1行の幅を14.2\,cm, 1ページあたり32行}とするこ
      と.日本文/英文にかかわらず,章・節の見出しは2行分とし,それ以
      下の小節の見出しは1行分とする.また箇条書の前後や項目間に余分な
      空白は挿入しないこと.
    \end{itemize}
\end{itemize}

\section{論文の構成}\label{sec-structure}

本節では特別研究報告書の構成について説明する.報告書には,内容梗概,目
次,本文を含まなければならない.必要であれば付録を加えることができる.

論文の書き方については,研究分野によるスタイルの違いも多く,本ガイドで
統一的な指針を提供するのが難しいところがある.指導教員のアドバイスを受
けつつ,各分野の論文を読み,何より多くの論文を自分で書くことで,スタイ
ルを身に付けていくことが必要である.なお,特別研究報告書のような自然科
学や科学技術に関する文章のまとめ方については,多くの良書がある.
%これらを一読されたい.
%\begin{itemize}
%  \item 
%\end{itemize}
%本節では,特別研究報告書において特に留意すべき点について述べる.

%% これらの区分を明確にすることと,それぞれの目的に従って記述すること
%% が必要である.

\subsection{内容梗概}\label{subsec-abstract}

内容梗概は,報告書の内容をまとめた短い文章である.内容梗概はそれ自身の
みを読んで報告書の内容が分かるように記述する必要がある.したがって,本
文を単に圧縮・要約するのではなく,研究の背景,目的,方法,得られた結論
が,その分野を専門とする研究者以外にもある程度分かりやすいようにまとめ
られている必要がある.

%% \footnote{内容梗概の多くの部分を後に述べる緒論
  %% (序論)が占める例がしばしば見られるが,内容梗概の目的からして不
  %% 適切であることはいうまでもない.}

特別研究報告書においては,前述の通り,報告書執筆に使用する用語(日本語
  または英語)にかかわらず,日本文と英文の内容梗概をそれぞれ1ページ半
から2ページにまとめることが定められている.専門外の研究者や技術者にも
伝わるように研究の内容を要約することは,それ自体が高度な知的作業である.
しっかり時間をかけて取り組まれたい.

\subsection{目次}\label{subsec-toc}

目次は一般の著書と同様の形式で,本文の直前に置く.これは単に各章や節が
どのページに書いてあるかを読者に知らせるためのみではなく,読者に論文全
体の構成や内容を伝えるために重要な情報となる.

\subsection{本文}\label{subsec-main}

本文は序論,本論,結論に分けて構成することが多い.

\subsubsection{序論}\label{subsubsec-intro}

序論は自分の研究が人間の知においてどのように位置づけられるかを読者に知
らせる目的がある.したがって,研究の文脈にあたる研究の歴史的背景や主要
な関連研究等の説明等から始めて,その研究で解決しようとする問題を読者に
理解させることが重要である.また,その研究がどのような方法で目的を達成
しようとするのか,どのような結果や貢献が得られたのかを説明することも,
序論の重要な目的である.

%% 研究の背景,目的,性格などを記述し,これによって,読む者にある程
%% 度の心構えを与え,その研究のおよその意義をあらかじめ理解させるように留
%% 意すべきである.

%% ただしこれは,上述の内容梗概とはおのずから目的が異なり,あくまで本論を
%% 読むための準備としての機能に重点を置くべきであって,問題の種類によって
%% は,歴史的な経過の記述,あるいは現状の展望を加え,また他の研究との関係
%% および相違点などにも触れることが必要である.

\subsubsection{本論}\label{subsubsec-main}

本論においては研究によってどのような成果が得られたかを説得的に記述する.
本論をどのような構成にするか,何を記述すべきかは,上記のように研究分野
によるスタイルの違いが大きい.しかし,どの分野においても,研究成果とし
て主張する貢献をサポートするのに必要かつ十分な内容が論理的に記述される
べきである.

%% 論文の主体であって,多くの紙数をこれに当てるのが普通である.また
%% 論理的に一貫した流れの中に重点が強調され,全体としてのまとまりが保たれ
%% るように工夫するとともに,引用と創意の区別を明らかにするなど,良心的な
%% 記述に心がけなければならない.

%% 前述のとおり,論文は分かりやすいことが第一の要件であるから,本論の記述
%% に当たっては,つぎの事項に注意する必要がある.
%% \begin{itemize}%{
%% \item
%% 適当な長さの章・節に分け,その順序,標題なども十分に検討する.必要ならば,節
%% をさらに細分し,適当な見出しをつける.
%% \item
%% 各章・節ごとに一応のまとまりをもつとともに,他の章・節への自然なつながりが保
%% たれるように留意する.
%% \item
%% だらだらとした表現を避け,記述の精疎の配分を工夫して,重要な点,独創的な部分
%% を強調する.
%% \item
%% 使用する記号の意味を正確に定義し,数式の誘導は,十分に整理された形で記載する.
%% 長い式の扱いは,全体を付録にまわし本論には重要な部分のみを書くほうが読みやす
%% くなる場合が多い.
%% \item
%% 近似式,実験式などは,その根拠を明示する.
%% \item
%% 図表はしばしば極めて重要な要素となるが,説明的な目的で入れるものと結論的な意
%% 味を持つものとの区別を明確にし,また引用した図表はその出典を明らかにする.
%% \item
%% 図はていねいに,正確に書き,図中の文字や記号,図表の見出し(標題または簡単な
%% 説明)にも十分な注意を払うこと.
%% \item
%% 引用文献は,研究に関係の深い重要なものを掲げ,無意味な羅列を避けること.文献
%% 表の挿入場所は本文の末尾(付録がある場合には,付録の前)とする.
%% \end{itemize}%}

\subsubsection{結論}\label{subsubsec-conclusion}

結論は論文のまとめとして,得られた研究成果を簡潔に述べる.序論において
も研究成果を簡潔にまとめることが多いが,これから論文を読もうとする読者
に伝えるべき研究成果の要約と,論文を一通り読んだ読者に伝えるべき研究成
果の要約とでは,含むべき情報がやや異なることが多い.研究途上に派生した
副次的な問題や将来に残された研究課題があれば,それらについても触れる.

%% 論文のまとめとして,研究成果の要点を簡潔に記述すべきである.これ
%% は当然,本論に置ける本質的な部分を圧縮したものとなるが,内容梗概とは異
%% なり,論文の締めくくりにふさわしい格調のうちに完結するように努めなけれ
%% ばならない.

%% また

\subsubsection{謝辞}\label{subsubsec-ack}

結論のあとに,研究上の指導,助言,援助を受けた人々に対して,謝辞を書く
慣習となっている.また(特別研究報告書においては必要ない場合も多いと思
  われるが)研究資金や研究機器を公的機関や民間企業から得ている場合は,
それらについて謝辞を書くことが必要な場合がある.

\subsection{付録}\label{subsec-appendix}

報告書は,本文のみで完結するようにまとめなければならないが,さらに本文
の内容を補足し,より充実したものとするために,本文のあとに付録を加える
ことができる.付録は,たとえばつぎのような場合に必要である.
\begin{itemize}%{
\item
  細かい証明や数式の変形等は,長さの関係で本文に記載できないことが多く,
  本文の可読性のためにも付録に含めることがある.(ただし,その定理の証
    明自体が研究の主目的である場合には,当然これを本文に入れるべきであ
    る.)この場合,本文および付録の両方に,互いの対応を明示しなければ
  ならない.
\item 研究成果の根拠となるデータや数値計算の結果などは,図表の形に整理
  したうえで本文に入れるべきであるが,その量が多いときには,参考資料と
  して付録に掲載する.
\end{itemize}

得られた生データや実験に用いたソースコードは通常は論文に含めない.これ
らを公開する場合には,適当なリポジトリ等に保存した上で,その URL 等を
論文中に含めることが多い.(これらを公開してよいか否かには,特許権や著
  作権等の知財やプライバシー等の問題を検討することが必要である.指導教
  員と必ず相談すること.)なお,公開しない場合も,研究成果の再現性の担
保のために,データやプログラムを研究室のサーバ等に保管しておくこと.

%% なども,長いものは付録に収める.
%% 特に大量のデータや長大なプログラムを添付したいときには,別冊付録としてもよい.
%% ただしこれも,保管,閲覧の便宜を考慮し,なるべく本文と同様の体裁にまとめるこ
%% とが望ましい.
%% 付録(特に別冊付録)には,適当な場所に標題と簡単な説明を付し,それだけで大体
%% の意味が分かるように配慮すべきである.


\section{報告書の記述に関する一般的なアドバイス}\label{sec:advice}

\subsection{術語}
術語に関しては,専門の学会誌等を参照して正確を期し,定訳のない術語は原
語のままとするか,原語を併記することが必要である.固有名詞は言語または
かたかなで書くが,かな書きの場合にも最初だけは原語を併記するのがよい.

\subsection{記号,単位}\label{subsec-symbol}
数式を多用する論文は,\LATEX などの数式を扱える組版ソフトで書くこと.
やむをえず通常のワードプロセッサを用いる場合や,図表などの中で数式を使
用する場合には,文字のフォント,サブスクリプトやスーパスクリプトの位置
などに十分な注意が必要である.

記号は全て明確に定義するべきである.多数の記号を使用する場合には,「記
  号表」を適当な場所に挿入する研究分野もある.

物理量の単位の略記法は,学会誌などで広く用いられている標準的なものに従
うべきであるが,標準化されていないものについては,説明を加える必要があ
る(脚注を利用してもよい).

\subsection{図表}\label{subsec-figure}
図表は\emph{全て本文中に挿入し},できるだけ本文で参照している箇所の近
くに配置する.

表の上側には表番号(たとえば表1.3)と簡単な見出しをつける.また図の下
側には図番号(たとえば図2.1)と簡単な見出し(必要ならば簡単な説明)を
つける.

図はできるだけ作図ツールなどを用いて電子的に作成すべきであるが,やむを
えず手書きで作成する場合には,いわゆる「版下」に使うつもりでていねいに
書かなければならない(鉛筆書きは許されない).

一般に,図表はそれをみただけで,およその意味が分かるように作成すること
が望ましい.また,本文にも対応する図表の番号が必ず現れるように注意しな
ければならない.

大量の観測データや計算結果に対する図表は,代表的なもののみを本文に入れ,
全体は付録や Web 上にまとめるほうがよい.

\subsection{脚注}\label{subsec-footnote}

脚注はむやみに挿入すべきではないとされているが,本文を分かりやすくする
ために,簡単な注釈を脚注として入れることは,場合によっては効果的である.

脚注と本文との対応は下の例のように,ページごとに付けた脚注番号による.

なお引用文献は,原則として参考文献リストの形にまとめるべきであるが,研
究の本題とあまり関係のない証明などの出所や,オープンソースソフトウェア
の URL を示す場合には,脚注を用いてもよい.

%% \begin{description}
%% \item[例{\dm (本文)}]\leavevmode\par \ldots この逐次近似法は,微分方
%%   程式の解の存在定理の証明に用いられ\footnote {この思想はPicardによっ
%%     て導入されたものである(1890).},\ldots, となることが知られてい
%%   る\footnote{この証明は,例えばWhittaker and Watoson: Modern
%%     Analysis, p.~123にみられる.}.
%% \end{description}

\subsection{文献}\label{subsec-references}

参考文献リストはタイトル,梗概,本文に並ぶ論文のもう一つの顔である.適
切に文献を引用できているかどうかで,研究成果をまとめて人類の知の体系に
位置づけるという論文執筆の目的の成否が決まることもある.引用すべき論文
はすべて引用し,不要な論文は引用しないようにすべきである.

\emph{参考文献リストの作成は BibTeX 等のツールを用いるべきである.手動
  で参考文献リストを作成すると,間違いなく誤りが含まれる.}執筆の早い
段階でこれらのツールの使用方法を調べ,使えるようにしておくこと.

%% 引用文献は本文の末尾に(付録があればそのまえに),表の形にまとめる.文
%% 献の参照は[1], [2]のように,[ ]つきの通し番号による.この番号は,該当
%% する文章の切れ目,または人名その他の単語に続いて挿入する.

%% 文献表には,番号に続いて,著者名,表題,雑誌名(または書名),巻,年号,ペー
%% ジなどを,この手引の「参考文献」にならって記載する.なお文献[1]$\sim$[6]は会
%% 議録や雑誌に収録された論文の例であり,文献[7]$\sim$[9]は単行本の例である.

%% 雑誌名の略記法は,学会によっても多少異なるが,慣用のものを用いてよい.ただし
%% 周知でないものは,むしろ雑誌名をそのまま書いたほうがよい.この手引の参考文献
%% の例では,[2]の``IEEE Trans. Computers''は``IEEE Transactions on Computers''
%% の略であり,[4]の「信学論(D)」は,「電子情報通信学会論文誌D」の略である.

%% 雑誌に対しては,その「巻」と「発行年」のほか,「号」および(または)「月」を
%% 入れたほうが検索に便利である.号と月の入れ方はつぎの例による.
%% \begin{eqnarray*}
%% &\hbox{第4巻,第10号,1995年,10月発行の雑誌}\\
%% &\Big\Downarrow\\
%% &\hbox{Vol.~4, No.~10 (Oct.~1995)}.
%% \end{eqnarray*}

\subsection{その他のアドバイス}\label{subsec-others}


\begin{itemize}
  \item 執筆した特別研究報告書は,将来的に国際会議やワークショップ等へ
    の投稿につながることが多い.これらの投稿時に一から書き直すことを避
    ける意味でも,できるだけ論文として通用するような報告書をまとめてほ
    しい.
  \item 提出までに,自分自身で何度も校正を重ね,論旨の飛躍や矛盾のない
    ように注意するとともに,よく文章を練り,誤字や誤記を除くように心が
    けなければならない.また,先輩に目を通してもらうことで,自分では気
    づきにくい不明瞭な点に気づくことができる.
\end{itemize}

\section{おわりに}\label{sec-conclusion}

この手引では,特別研究報告書をどのような構成とするか,またどのような形
式で作成するかを説明した.しかし最初にも述べたように,手引に従いさえす
ればよい論文が書けるというものではない.

最も大切なのは,自分の研究成果を読者に理解してもらおうとする意欲と,論
文を少しでも優れたものにしようとする熱意である.この意欲と熱意とを持っ
て,各自が成し遂げた研究を締めくくられんことを願う.

なお,報告書および論文の作成に関して不明の点があれば,指導教員に相談されたい.

\acknowledgments				% 謝辞
本手引の作成にご協力頂いた,計算機科学コースの教員各位に深甚の謝意を表する.

\nocite{*}
\bibliographystyle{kuisunsrt}			% 文献スタイルの指定
\bibliography{kuisthesis_jp}				% 参考文献の出力

						% 付録の開始
\Appendix[付録:スタイルファイル{\tt kuistheis}の使用法]
この手引で述べた教室所定の形式に適合した論文を\LaTeX で作成するために,スタ
イルファイル\|kuisthesis|が用意されている.以下,\|kuisthesis|を使う
ための準備と,その使用法について解説する.
なお,この手引自体も\|kuisthesis|を用いて作成したものであるので,必要に
応じてスタイルファイルとともに配布されるソースファイルを参照するとよい.
また,論文作成の際に使用する\LaTeX コマンドのほとんどは標準的なものであるの
で,基本的な使用法やここで解説していないものについては以下の書籍等を適宜参照されたい.
\begin{quote}%{
Lamport, L.: {\em A Document Preparation System {\LaTeX} User's Guide \&
Reference Manual\/}, Addison Wesley, Reading, Massachusetts (1986).
(Cooke, E., et al.訳:文書処理システム{\LaTeX}, アスキー出版局(1990)).
\end{quote}%}

\section{ソースファイルの構成}\label{app-structure}
ソースファイルは以下の形式で作る.
\begin{itemize}\item[]%{
\|\documentclass{kuisthesis}|または\\
\|\documentclass[english]{kuisthesis}|\\
必要ならば他のオプションやスタイルファイルを指定する.\\
必要ならばユーザのマクロ定義などをここに書く.\\
\|\jtitle{|\<題目(和文)\>\|}|\\
\|\etitle{|\<題目(英文)\>\|}|\\
\|\jauthor{|\<著者名(和文)\>\|}|\\
\|\eauthor{|\<著者名(英文)\>\|}|\\
\|\supervisor{|\<指導教官名\>\|}|\\
\|\date{|\<提出年月日\>\|}|\\
%\|\department{|\<専攻名\>\|}|\\
\|\begin{document}|\\
\|\maketitle|\\
\|\begin{jabstract}|\\
\null\qquad\<内容梗概(和文)\>\\
\|\end{jabstract}|\\
\|\begin{eabstract}|\\
\null\qquad\<内容梗概(英文)\>\\
\|\end{eabstract}|\\
\|\tableofcontents|\hfill\rlap{\hskip-.5\linewidth{\tt\%}目次の出力}\\
\|\section{|\<第1章の表題\>\|}|\\
\null\qquad\hbox to3em{\dotfill}\\
\null\qquad\<本文\>\\
\null\qquad\hbox to3em{\dotfill}\\
\|\acknowledgments|\\
\null\qquad\<謝辞\>\\
\|\bibliographystyle{kuisunsrt}|\quad または\\
\|\bibliographystyle{kuissort}|\\
\|\bibliography{|\<文献データベース\>\|}|\\
付録があれば \|\appendix|/\|\Appendix| に続いてここに記す.\\
\|\end{document}|
\end{itemize}%}

\subsection{印字の形式}\label{appsub-format}
論文の各ページは,幅(\|\textwidth|) 14.2\,cm, 高さ(\|\textheight|) 
22.2\,cmの領域に印刷される\footnote{NTT版では和文の場合,幅 
({\tt\string\textwidth})が 13.6\,cmとなる}.この幅は和文の場合には35文字分に
相当し,高さは和文/英文とも32行分に相当するので,\ref{sec-instruction}章に
示した基準に合致している.
和文/英文とも\|\normalsize|のフォントは12\,ptであり,これも
\ref{sec-instruction}章の基準を満たしている.

\subsection{オプション・スタイル}\label{appsub-option}
\|\documentclass|の標準オプションとして,以下が用意されている.
\begin{itemize}%{
%\item
%\|master|\\
%修士論文用.指定がなければ特別研究報告用となる.なお両者の違いは,とびらに印
%字される論文種別と所属のみであり,ページ数のチェックなどは一切行なわない.
\item
\|english|\\
英文用.指定がなければ和文用となる.
%特別研究報告書は必ず和文であるので, 
%\|master|を指定せずに\|english|を指定するのは誤りであるが,特にチェックはしない.
\item
\|withinsec|\\
図表番号や数式番号を,``\<章番号\>.\<章内番号\>''の形式とする.指定がなけれ
ば,論文全体で通し番号となる.
\end{itemize}%}

この他に,\|epsf|など補助的なスタイルファイルを指定してもよい.ただしスタイ
ルファイルによっては,論文スタイルと矛盾するようなものもあるので,スタイルファ
イルの性格をよく理解して使用すること.たとえば,\|a4|はページの高さである
\|\textheight|を変更するので,使用してはならない.

\subsection{題目などの記述}\label{appsub-title}
論文の題目,著者名,および指導教官名を前に示した所定のコマンドで指定した後,
\|\maketitle|を実行すると,とびらが生成される.
とびらのページにはページ番号が印字されないが,出力の便宜を図るためにdviファ
イルにはページ番号1000が付与されている.
とびらには,以下の項目がそれぞれセンタリングされて,順に印字される.
\begin{description}%{
\item[論文種別]
\|\documentclass|のオプションにしたがって,
「特別研究報告」または``Graduation Thesis''のいずれかが\|\Large\bf|で印字される.

\item[題目]
和文の場合には\|\jtitle|で,英文の場合には\|\etitle|で指定した題目が,それぞ
れ\|\LARGE\bf|で印字される.一行に収まらない場合には自動的に改行されるが,適
切な箇所に\|\\|を挿入して陽に改行を指示するほうがよい.

\|\jtitle|や\|\etitle|で指定した題目は,とびらだけではなく内容梗概や目次にも
印字される.したがって和文/英文に関わらず,\|\jtitle|と\|\etitle|の双方を指
定しなければならない.また,とびらと内容梗概/目次では,題目の改行を違う位置
で行ないたいこともあるだろう.その場合
\begin{quote}%{
\|\jtitle[|\<内容梗概/目次用\>\|]{|\<とびら用\>\|}|\\
\|\etitle[|\<内容梗概/目次用\>\|]{|\<とびら用\>\|}|
\end{quote}%}
のように,オプション引数で内容梗概や目次のページに印字する題目を別途指定することができる.

\item[指導教員名]
\|\supervisor|で指定した指導教員の氏名と職名を\|\large|で印字する.氏名/職
名は,本文に用いる言語に応じて適切に指定すること.

\item[所属学科]
\|\documentclass|のオプションに応じて,以下のいずれかが\|\large|で印字される.
\begin{itemize}%{
\item 特別研究報告書\\
京都大学工学部情報学科
\item Graduation Thesis\\
School of Informatics and Mathematical Science\\
Faculty of Engineering\\
Kyoto University
\end{itemize}

\item[著者名]
和文の場合には\|\jauthor|で,英文の場合には\|\eauthor|で指定した著者名が,そ
れぞれ\|\Large|で印字される.題目と同様,\|\jauthor|と\|\eauthor|は内容梗概
のページにも印字されるので,和文/英文に関わらず双方を指定すること.

\item[提出年月日]
\|\date|で指定した日付が\|\large|で印字される.日付は,本文に用いる言語に応
じて適切に指定すること.

%\item[専攻名]
%修士論文の場合、\|\department|で指定した専攻名が印字される.例えば、
%和文の場合
%\begin{quote}\begin{verbatim}
%\department{社会情報学}
%\end{verbatim}\end{quote}
%英文の場合
%\begin{quote}\begin{verbatim}
%\department{Social Informatics}
%\end{verbatim}\end{quote}
%のように指定する.
\end{description}

\subsection{内容梗概}\label{appsub-abstract}
和文の内容梗概を\|jabstract|環境の中に,また英文の内容梗概を\|eabstract|環境
の中に,それぞれ記述する.それぞれの内容梗概の前には,前述の\|\jtitle|や
\|\etitle|で指定した題目と,\|\jauthor|や\|\eauthor|で指定した著者名が出
力される.

それぞれの内容梗概は,記述した順序で出力される.したがって,本文が和文の場合
には和文\,$\to$\,英文の順で,また本文が英文の場合には英文\,$\to$\,和文の順で
記述するのが適当である.

内容梗概のページ番号は,ページの右肩に小文字のローマ数字で印字される.また出
力の便宜を図るために,dviファイルの各ページには印字されるページ番号に1000を
加えたものが付与される.

\subsection{目次}\label{appsub-toc}
コマンド\|\tableofcontents|により,目次が生成される.目次の最上部には,前述
の\|\jtitle|\slash\|\etitle|で指定した題目が印字される.

デフォルトでは,\|\section|, \|\subsection|, および\|\subsubsection|の見出し
とそれらのページ番号が目次に含まれる.これを変更し,たとえば\|\section|と
\|\subsection|のみの目次にしたい時には
\begin{quote}\begin{verbatim}
\setcounter{tocdepth}{2}
\end{verbatim}\end{quote}
により,カウンタ\|tocdepth|の値を目次に含まれる最下位の章・節レベル
に設定すればよい.なお\|\section|のレベルは1である.

この他,「謝辞」と「参考文献」も番号のない\|\section|として目次に含まれる.
さらに(もしあれば)「付録」と,付録の中の\|\section|と\|\subsection|も含ま
れる.

目次のページにはページ番号を印字しないが,dviファイルには内容梗概に続く1000
番台のページ番号が付与される.

\subsection{章・節}\label{appsub-sectioning}
章や節の見出しには,通常どおり\|\section|, \|\subsection|, \|\subsubsection|
などを使用する.

\|\section|の見出しは2行を占め,\|\Large\bf|で印字される.修士論文の
場合は改頁が行なわれる.
\|\subsection|の見出しは1行の空白を置いた後に\|\large\bf|で印字され,引
き続く文章との間には余分な空白は挿入されない.\|\subsubsection|は
上部に空白が挿入されず、\|\normalsize\bf|で印字される.

デフォルトでは,上記の3つのコマンドによる章・節の見出しに,章番号や節番号が
付けられ,下位の章・節コマンドである\|\paragraph|, \|\subparagraph|による見
出しには番号が付けられない.また,これらの下位コマンドによる見出しと引き続く
文章の間では改行が行なわれない.

\subsection{図表}\label{appsub-figure}
図や表は,通常と同じく\|figure|や\|table|環境の中に記述する.図表の番号は,
デフォルトでは論文全体の通し番号であるが,前述の\|\documentclass|のオプショ
ン\|withinsec|を使用すると,章の中で番号づけが行なわれ,章番号と組み合わされ
る.

紙面の節約のために,図表を横に並べて置きたいことがある.このような場合のため
に,\|subfigure|と\|subtable|という2つの環境が用意されている.たとえば図
\ref{fig-example}と表\ref{tab-example}は
\begin{quote}%{
\|\begin{figure}|\\
\|\begin{subfigure}{0.6\textwidth}|\\
\null\qquad\<図\ref{fig-example}の中身\>\\
\|\caption{図の例}|\\
\|\end{subfigure}|\\
\|\begin{subtable}{0.4\textwidth}|\\
\|\caption{表の例}|\\
\null\qquad\<表\ref{tab-example}の中身\>\\
\|\end{subtable}|\\
\|\end{figure}|
\end{quote}%}
により生成したものである.なおこの例では\|figure|環境の中に\|subfigure|と
\|subtable|を入れているが,\|table|環境の中に入れてもよい.

\|subfigure|と\|subtable|の仕様は,\|minipage|と同様であり
\begin{itemize}\item[]%{
\|\begin{subfigure}[|\<位置\>\|]{|\<横幅\>\|}|\quad\<中身\>\quad
\|\end{subfigure}|\\
\|\begin{subtable}[|\<位置\>\|]{|\<横幅\>\|}|\quad\<中身\>\quad
\|\end{subtable}|
\end{itemize}%}
である.また環境中の\|\caption|コマンドにより,それぞれの見出しが生成される.

横に並べる\|subfigure|\slash\|subtable|の横幅の合計が,
\|\textwidth|に一致するようにするのが望ましい\footnote{各々の間に
{\tt\string\hspace\char`\{\string\fill\char`\}}を挿入するなどして,間隔を置く
こともできる.}.

\begin{figure}%{
\begin{subfigure}{.6\textwidth}
\centerline{\fbox{\vbox to.1\textheight{\vss
	\hbox to.8\textwidth{\hss This is a figure\hss}\vss}}}
\caption{図の例}\label{fig-example}
\end{subfigure}
\begin{subtable}{.4\textwidth}
\caption{表の例}\label{tab-example}
\centerline{\begin{tabular}{r|c|l}
This&is&a table\\\hline
placed&beside&a figure.
\end{tabular}}
\end{subtable}
\end{figure}%}

\subsection{箇条書}\label{appsub-itemizing}
\LaTeX の箇条書環境である\|\enumerate|, \|\itemize|, \|\description|などは,
すべてそのまま使用することができる.ただし,環境の前後や,項目の間には余分な
空白が挿入されない.

\subsection{脚注}\label{appsub-footnote}
脚注には,\LaTeX の標準コマンド\|\footnote|を用いる.脚注のマークは,ここに
示すように\footnote{脚注の例}や\footnote{もう一つの脚注}である.また,このペー
ジと前のページを見るとわかるように,脚注番号はページごとに付けられる.ただし,
正しい脚注番号を得るためには,\LaTeX を2回実行する必要がある.

\subsection{謝辞}\label{appsub-acknowlegments}
謝辞は,コマンド\|\acknowledgments| に続いて記述する.見出し「謝辞」または
``Acknowledgments''は自動的に生成され,目次にも登録される.

\subsection{参考文献}\label{appsub-references}
すべての参考文献を含むようなBib\TeX の文献データベースを作成し,文献スタイル
ファイル\|kuisunsrt|または\|kuissort|を用いて処理すれば,
\ref{subsec-references}節に示した形式の文献表が得られる.なお\|kuisunsrt|は文
献を出現順に並べ,\|kuissort|は著者名のアルファベット順に並べる.

何らかの理由でBib\TeX を利用できない場合は,\|thebibliography|環境を用いて文
献表を作ってもよいが,この手引の文献表を参考にして指定された形式に従うこと.

なお,いずれの場合にも,見出し「参考文献」または``References''が自動的に生成
され,目次にも登録される.

\subsection{付録}\label{appsub-appendix}
付録がもしあれば,コマンド\|\appendix|または\|\Appendix|に引き続いて記述する.
両者の違いはページ付けであり,\|\appendix|では付録の各ページや目次にページ
番号が印字されない.一方\|\Appendix|では,付録の先頭ページをA-1とし,順
にA-2, A-3というページ番号が印字される.なおいずれの場合にも,dviファイルに
は2001から始まるページ番号が付与される.

どちらのコマンドもオプション引数を持ち,付録全体の見出しをつけることができる.
たとえば,この付録は
\begin{quote}
\|\Appendix[付録:スタイルファイル{\tt kuisthesis}の使い方]|
\end{quote}
で始まっている.オプション引数がない場合には,付録全体の見出しは単に「付録」
または``Appendix''である.

付録の中の\|\section|, \|\subsection|などは,1レベル下のコマンドと同じ動作を
する.またこれらの番号は,``A.1''や``A.2.3''のように,先頭に``A.''が付加され
たものとなる.同様に,図表や数式の番号にも,先頭に``A.''が付加される
\footnote{この番号付けは{\tt\string\documentclass}の{\tt withinsec}オプションと
は無関係である.}.

\section{その他の注意}\label{app-others}
\LaTeX の大きな特徴の一つは,文書処理に関するさまざまな機能やパラメータをカ
スタマイズできることである.したがって,少しでも論文を書きやすくするために,
学生諸君の創意と工夫で個人用の機能を追加したりするのはもちろん自由であり,む
しろ推奨される.しかし一方では,教室で定められた形式を守ることも必要であり,
カスタマイズの際にはこの点に注意しなければならない.

どのようなカスタマイズが許されるかを一般的に述べるのは困難であるが,一つの極
端な基準は,スタイルファイルを読んでみて大丈夫だと確信が持てること以外はしな
い,というものである.特に\LaTeX{}nicianであるような諸君には,この基準を厳守
してもらいたい.

一般の学生諸君のためのもう少し緩やかな基準として,コマンドやパラメータの再定
義/再設定を行なわない,というものも挙げられる.スタイルファイルを読むのが面
倒だったり,読んでもよくわからなかったりする場合には,この基準を守ってもらい
たい.

スタイルファイルの作成に当たっては,バグがないように細心の注意を払っているが,
% 適用例が少ないこともあり,
完璧なものとなっているとは断言できない.
% もし何か問
% 題が起こった場合には,教室のローカル・ニュース・グループ\|is.misc| に投
% 稿されたい.またスタイルファイルの改版などの通知も,同じニュース・グループに
% 投稿されるので,注意しておくこと.なお,担当教員などへの直接の質問には一切応
% じない.
\end{document}
